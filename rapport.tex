\documentclass[11pt]{article}
\usepackage[a4paper]{geometry}
\usepackage{mathtools}
\usepackage{amssymb,amsthm,amsfonts}
\usepackage{titling}
\usepackage{polyglossia}
\setdefaultlanguage{french}

\DeclareMathOperator{\id}{id}
\DeclareMathOperator{\Ker}{Ker}
\DeclareMathOperator{\Img}{Im}
\DeclareMathOperator{\Mat}{Mat}
\DeclareMathOperator{\Tr}{Tr}
\DeclareMathOperator{\rank}{rg}
\DeclareMathOperator{\vect}{Vect}
\DeclareMathOperator{\diagmat}{Diag}
\newcommand{\gl}{\mathrm{GL}}
\newcommand{\mev}[1]{\mathcal{M}_{#1}}
\newcommand{\dif}[1][{}]{\mathop{}\!\mathrm{d}^{#1}}

\newcommand{\ZZ}{\mathbb{Z}}
\newcommand{\QQ}{\mathbb{Q}}
\newcommand{\RR}{\mathbb{R}}
\newcommand{\CC}{\mathbb{C}}
\newcommand{\KK}{\mathbb{K}}
\newcommand{\PP}{\mathbb{P}}
\DeclarePairedDelimiter{\zintcc}{[\![}{]\!]}

%%% Fonts %%%
\setmainfont{XITS}

%%% Titling %%%

\pretitle{\hrulefill\begin{center}\LARGE}
\title{
	\textbf{Rapport de projet MAP411}:\\
	\textit{Résolution des problèmes elliptiques symétriques en grande dimension}}
\posttitle{\end{center}\hrulefill}


\author{
	\textit{Cheikh Fall}\\
	\textit{Wilson Jallet}\\
Promotion X2016}

%%% Questions %%%

\theoremstyle{definition}
\newtheorem{ques}{Question}

\setlength{\parindent}{0pt}

\begin{document}
\maketitle

\section{L'algorithme glouton}

\setlength{\parindent}{1cm}
\setlength{\parskip}{3mm}

\begin{ques}
On rappelle l'équation de Laplace
\begin{equation}\label{eqLapl}
\left\{
\begin{array}{c}
-\Delta u + u = f\ \text{sur}\ \Omega \\
\dfrac{\partial u}{\partial n} = 0\ \text{sur}\ \partial\Omega
\end{array}
\right.
\end{equation}
où $u \in C^2(\overline{\Omega})$, $\Omega = (0,1)^2$.

Supposons que la fonction $u:\overline\Omega\longrightarrow\RR$ soit une solution de \eqref{eqLapl}. Soit $v\in V$. Par une intégration par parties, on obtient
\begin{align*}
-\int_\Omega \Delta u\, v &= -\int_{\partial\Omega} \frac{\partial u}{\partial n}v + \int_\Omega \nabla u\cdot \nabla v \\
&= \int_\Omega \nabla u\cdot \nabla v \quad \text{d'après \eqref{eqLapl}}
\end{align*}
puis en utilisant que $-\Delta u = f-u$
\[
	-\int_\Omega uv + \int_\Omega fv = \int_\Omega \nabla u\cdot \nabla v
\]
et
\begin{equation}\label{laplVar}
\forall v\in V\quad
\int_\Omega \nabla u\cdot \nabla v + \int_\Omega uv = \int_\Omega fv.
\end{equation}


Réciproquement, si $u$ est une fonction de classe $C^2$ sur $\overline\Omega$ qui vérifie \eqref{laplVar}, une intégration par parties permet de retrouver 
\[
\forall v\in V\quad 
\int_{\partial\Omega} \frac{\partial u}{\partial n}v -\int_\Omega \Delta u\, v + \int_\Omega uv = \int_\Omega fv
\]
En particulier $-\int_\Omega \Delta u\,v + \int_\Omega uv = \int_\Omega fv $ quand $v=0$ sur $\partial\Omega$. D'après un lemme du cours, cela entraîne que\[ -\Delta u+ u -f = 0  \]
et en réinjectant dans l'équation intégrale plus haut
\[
\forall v\in V\quad 
\int_{\partial\Omega}\frac{\partial u}{\partial n}v = 0\quad \text{d'où}\quad \frac{\partial u}{\partial n} = 0
\]
et $u$ est solution de \eqref{eqLapl}.

\end{ques}


\begin{ques}
Pour tous $i$ et $j$ de $\zintcc{0,I}$,
\[
\nabla \phi_i\otimes\phi_j (x,y) = \left(
\phi_i'(x)\phi_j(y), \phi_i(x)\phi_j'(y)
\right)
\]
donc 
\begin{align*}
\int_{\Omega}\nabla(\phi_i\otimes\phi_j)\cdot\nabla(\phi_k\otimes\phi_l) 
&=
\int_{\Omega} \phi'_i(x)\phi_j(y)\phi_k'(x)\phi_l(y) + \phi_i(x)\phi_j'(y)\phi_k(x)\phi_l'(y)\,dx\,dy \\
&= \int_{0}^{1}\phi'_i\phi_k' \int_0^1\phi_j\phi_l + \int_{0}^{1}\phi_i\phi_k \int_0^1\phi_j'\phi_l' \ \text{(théorème de Fubini)} \\
&= D_{i,k}M_{j,l} + M_{i,k}D_{j,l}.
\end{align*}

Ainsi, avec $u_h = \sum_{i,j=0}^{I}U_{i,j}\phi_i\otimes\phi_j$ on obtient
\begin{align*}\tag{a}\label{eq:a}
\int_{\Omega} \nabla u_h(x,y)\cdot \nabla \phi_k\otimes\phi_l (x,y)\,dx\,dy 
&= \sum_{i,j=0}^{I}U_{i,j} \int_\Omega\nabla(\phi_i\otimes\phi_j)\cdot\nabla(\phi_k\otimes\phi_l) \\
&= \sum_{i,j=0}^{I}U_{i,j} (D_{i,k}M_{j,l}+M_{i,k}D_{j,l}).
\end{align*}
On a de plus
\[
\int_\Omega (\phi_i\otimes\phi_j)(x,y)(\phi_k\otimes\phi_l)(x,y)\,dx\,dy = \int_0^1 \phi_i \phi_k \int_0^1 \phi_j \phi_l = M_{i,k}M_{k,l}
\]
d'où
\begin{equation}\tag{b}\label{eq:b}
\int_\Omega u_h(x,y)\phi_k\otimes\phi_l = \sum_{i,j=0}^{I}U_{i,j}M_{i,k}M_{j,l}
\end{equation}

En supposant que $u_h$ est une solution de \eqref{laplVar}, sommer \eqref{eq:a} et \eqref{eq:b} donne pour tous $k,l\in\zintcc{0,I}$
\begin{equation}\label{laplDiscr}
\sum_{i,j=0}^{I}U_{i,j}(D_{i,k}M_{j,l} + M_{i,k}D_{j,l} + M_{i,k}M_{j,l}) =
\int_\Omega\nabla u_h\cdot \nabla\phi_k\otimes\phi_l + \int_\Omega u_h (\phi_k\otimes\phi_l)
= F_{k,l}.
\end{equation}
On notera cela sous la forme $TU_h = F$ dans la suite.

Réciproquement, si le vecteur $U$ vérifie le système linéaire \eqref{laplDiscr}, alors la fonction $u_h = \sum_{i,j=0}^{I}U_{i,j}\phi_i\otimes\phi_j$ vérifie \eqref{laplVar} pour $v\in V_h$.

Le problème \eqref{laplDiscr} est bien posé puisque \eqref{laplVar} l'est: si $u_1$ et $u_2$ vérifient \eqref{laplVar}, alors $w:= u_2-u_1$ vérifie
\[
\int_\Omega \left|\nabla w\right|^2 + \int_\Omega w^2 = \|w\|_{H^1(\Omega)}^2 = 0
\]
donc $w = 0$, $u_2 = u_1$.

Si on travaille sur un pavé $\Omega = (0,1)^d$ en dimension $d$ avec la même méthode, la base tensorielle est $\phi_{i_1}\otimes\cdots\otimes\phi_{i_d}$ et on travaille avec des données $U \in \RR^{(I+1)^d}$.

\end{ques}


\begin{ques}
Si $U_h\in \RR^{(I+1)^2}$ et $u_h = \sum_{i,j=0}^I U_{i,j}\phi_i\otimes\phi_j$, remarquons qu'avec la notation introduite plus haut
\[
\mathcal{E}(u_h) = \frac 12\int_\Omega |\nabla u_h|^2 + \frac 12\int_\Omega u_h^2 - \int_\Omega fu_h = \frac 12\langle TU_h,U_h\rangle - \langle U_h,F\rangle.
\]

Le membre de droite est une fonction de classe $C^1$ sur $\RR^{(I+1)^2}$ dont le gradient est $TU_h - F$, sa dérivée seconde est l'application bilinéaire définie positive
\[
(X,Y)\in \RR^{(I+1)^2}\times \RR^{(I+1)^2} \longmapsto \langle TX,Y\rangle
\]
On en tire que $U_h$ est solution de \eqref{laplDiscr} si et seulement si c'est un minimum du membre de droite, \fbox{si et seulement si $u_h$ est un minimum de $\mathcal{E}$ restreint à $V_h\otimes V_h$:}
\begin{equation}
u_h = \sum_{i,j=0}^I U_{i,j}\phi_i\otimes\phi_j = \arg\min_{v_h\in V_h\otimes V_h}\mathcal{E}(v_h)
\end{equation}
\end{ques}


\begin{ques}
L'algorithme est glouton puisqu'il consiste à faire une optimisation locale à chaque étape afin de trouver ce qui sera un optimum local de la fonction $\mathcal E$.

Si $N$ est le numéro de la dernière itération, il faut stocker les en mémoire les fonctions $r_n,i\in V_h$, $i\in\zintcc{1,d}$ et $1\leq n\leq N$, ce qui donne, en supposant les éléments de $V_h$ représentés selon leur coordonnées avec $I+1$ réels, $(I+1)Nd$ nombres réels à stocker.

Supposons une représentation séparée de la fonction $f$
\[
f(x_1,\ldots,x_d) = \sum_{p=1}^{P} f_1^p(x_1)\ldots f_d^p(x_d)
\]
Alors, les coefficients du vecteur $F = (\int_\Omega f\phi_{i_1}\otimes\cdots\otimes\phi_{i_d})$ s'écrivent grâce au théorème de Fubini
\[
F_{i_1,\ldots,i_d} = \sum_{p=1}^P\prod_{j=1}^{d}\int_0^1 f_j^p(x)\phi_{i_j}(x)\,dx
\]
et pour les obtenir, il suffit de calculer et stocker les $Pd(I+1)$ intégrales
\[
\int_0^1 f_j^p\phi_i,\quad j\in\zintcc{1,d},\,i\in\zintcc{0,I},\,p\in\zintcc{1,P}.
\]

Si $f$ n'admet pas de représentation séparée, il faut alors calculer et stocker les $(I+1)^d$ intégrales $\int_\Omega f\phi_{i_1}\otimes\cdots\otimes\phi_{i_d}$, ce qui fait nettement plus de données.
\end{ques}

\section{Équations d'Euler et convergence de l'algorithme}

On considère le problème de minimisation
\begin{equation}
(r_n,s_n) \in \arg\min_{(r,s)\in V_h\times V_h} \mathcal{E}(u_{n-1}+r\otimes s).
\end{equation}


\begin{ques}
On a montré plus haut que la restriction de l'énergie $\mathcal{E}:C^1(\overline{\Omega})\longrightarrow \RR$ à l'espace $V_h\otimes V_h$ admet un minimum. Il en est donc de même de $\mathcal J:(r,s)\in V_h\times V_h\longmapsto \mathcal{E}(u_{n-1} + r\otimes s)$ quel que soit $u$ dans $V$ (puisque $r\otimes s\in V_h\otimes V_h$), et $(r,s)$ en est un minimum si et seulement si c'en est un point critique.

La différentielle de $\mathcal{J}$ est
\begin{align*}
d_{(r,s)}\mathcal J(\delta r,\delta s) 
&= d\mathcal{E}_{u_{n-1}+r\otimes s}\cdot
\underbrace{\left(
d_{(r,s)}(u_{n-1}+r\otimes s)(\delta r,\delta s)\right)
}_{= \delta r\otimes s + r\otimes\delta s} \\
&= \int_\Omega \nabla(u_{n-1}+r\otimes s)\cdot\nabla(\delta r\otimes s + r\otimes\delta s)
+ \int_\Omega (u_{n-1}+r\otimes s)(\delta r\otimes s + r\otimes\delta s)
\\ &\quad - \int_\Omega f(\delta r\otimes s + r\otimes\delta s).
\end{align*}

On en déduit que $(r_n,s_n)$ est un point critique de $\mathcal{J}$ si et seulement si $d_{(r_n,s_n)}\mathcal{J}(\delta r,\delta s) = 0$ pour tout $(\delta r,\delta s)\in V_h\times V_h$, soit
\begin{align}\label{eq:6}
\int_\Omega &\nabla(r_n\otimes s_n)\cdot\nabla(\delta r\otimes s_n + r_n\otimes \delta s) + \int_\Omega (r_n\otimes s_n)(\delta r\otimes s_n + r_n\otimes \delta s) \nonumber \\
&=\int_\Omega f(\delta r\otimes s_n + r_n\otimes\delta s) - \int_\Omega \nabla u_{n-1}\cdot\nabla(\delta r\otimes s_n + r_n\otimes\delta s) - \int_\Omega u_{n-1}(\delta r\otimes s_n + r_n\otimes\delta s)
\end{align}

Réécrire cela en $\delta r=0$ et $\delta s=0$ et intégrer par parties mène au système d'équations différentielles couplées
\begin{align*}
-r_n''(x)\int_0^1 s_n^2 + r_n(x)\int_{0}^{1}{s_n'}^2 &= \int_{0}^{1}(f(x,y)+\Delta u_{n-1}(x,y) - u_{n-1}(x,y))s_n(y)\,dy \\
-s_n''(y)\int_0^1 r_n^2 + s_n(y)\int_{0}^{1}{r_n'}^2 &= \int_{0}^{1}(f(x,y)+\Delta u_{n-1}(x,y) - u_{n-1}(x,y))r_n(x)\,dx
\end{align*}


\end{ques}


\begin{ques}
Comme $u_h$ est solution du problème variationnel \eqref{laplDiscr}, nous avons
\[
a(u_h,\delta r\otimes s_n + r_n\otimes \delta s) = \int_\Omega f(\delta r\otimes s_n + r_n\otimes \delta s)
\]
d'une part. D'autre part, remarquons que la différentielle de $\mathcal{E}$ se réécrit
\[
d\mathcal{E}_u(v) = a(u,v) - \int_\Omega fv
\]
donc la condition d'extremum portant sur $(r_n,s_n)$ se réécrit, en remarquant que $u_n = u_{n-1} + r_n\otimes s_n$
\begin{align*}
a(u_n,\delta r\otimes s_n + r_n\otimes\delta s) &=
a(u_{n-1}+r_n\otimes s_n, \delta r\otimes s_n + r_n\otimes\delta s) \\
&= \int_\Omega f(\delta r\otimes s_n + r_n\otimes\delta s).
\end{align*}
Ainsi,
\[
\boxed{a(g_n,\delta r\otimes s_n + r_n\otimes\delta s) = a(u_h-u_n,\delta r\otimes s_n + r_n\otimes\delta s) = 0.}
\]
En particulier, avec $\delta r = r_n$ et $\delta s= s_n$
\[
2a(g_n,r_n\otimes s_n) = 0.
\]

Remarquons que $g_n + r_n\otimes s_n = u_h - (u_n - r_n\otimes s_n) = u_h - u_{n-1} = g_{n-1}$, d'où par bilinéarité et symétrie de $a$
\begin{align*}
\boxed{a(g_{n-1},g_{n-1})} 
&= a(g_n,g_n) + \underbrace{2a(g_n,r_n\otimes s_n)}_{=0} + a(r_n\otimes s_n,r_n\otimes s_n) \\
&\boxed{= a(g_n,g_n) + a(r_n\otimes s_n,r_n\otimes s_n).}
\end{align*}

La remarque plus haut permet d'écrire
\begin{align*}
\boxed{E_n = \mathcal{E}(u_n) - \mathcal{E}(u_{n-1})}
&= \frac 12a(u_n,u_n) - \frac 12 a(u_{n-1},u_{n-1}) - \int_\Omega f(r_n\otimes s_n) \\
&= \frac 12 a(r_n\otimes s_n,r_n\otimes s_n) + a(u_{n-1},r_n\otimes s_n) - \int_\Omega f(r_n\otimes s_n) \\
&\boxed{= \frac 12\int_\Omega |\nabla(r_n\otimes s_n)|^2 + \frac 12\int_\Omega |r_n\otimes s_n|^2 - \int_\Omega f(r_n\otimes s_n) + a(u_{n-1},r_n\otimes s_n).}\tag{d}\label{eq:d}
\end{align*}

Il est immédiat que les deux premiers termes du membre de droite valent $\frac 12a(r_n\otimes s_n)$.

Remarquons que le membre de gauche de \eqref{eq:6} est $2a(r_n\otimes s_n,r_n\otimes s_n)$ en $\delta r=r_n$, $\delta s=s_n$. Les deux derniers termes du membre de droite de \eqref{eq:d} vérifient donc:
\begin{align*}
-\int_\Omega f(r_n\otimes s_n) + a(u_{n-1},r_n\otimes s_n) 
&= -\int_\Omega f(r_n\otimes s_n) + \int_\Omega \nabla u_{n-1}\cdot\nabla(r_n\otimes s_n) + \int_\Omega u_{n-1}(r_n\otimes s_n) \\
&= -a(r_n\otimes s_n,r_n\otimes s_n)
\end{align*}

On en déduit donc que \eqref{eq:d} se réécrit
\[
\boxed{E_n = -\frac 12a(r_n\otimes s_n,r_n \otimes s_n).}
\]

On a montré plus haut que $a(g_{n-1},g_{n-1}) = a(g_n,g_n) + a(r_n\otimes s_n,r_n\otimes s_n)$ ce qui entraîne puisque $a(r_n\otimes s_n,r_n\otimes s_n)\geq 0$ que la suite de réels positifs $\left(a(g_n,g_n)\right)_n$ est décroissante, donc converge. On en tire que la série télescopique
\(
\sum_{n\geq 1} \left(a(g_{n-1},g_{n-1}) - a(g_n,g_n) \right)
\)
converge. Or c'est exactement la série
\[
\sum_{n\geq 1} a(r_n\otimes s_n,r_n\otimes s_n),
\]
dont on déduit la convergence, puis celle de la série $-2\sum_{n\geq 1}E_n$ et l'égalité de leurs sommes d'après l'identité plus haut.
\end{ques}

\begin{ques}
Le raisonnement précédent montre que la suite $(g_n)_{n\geq 1}$ de $V_h\otimes V_h$ est bornée en norme $H^1$ (la norme $v\longmapsto \sqrt{a(v,v)}$). Notons que cette norme équipe l'espace de dimension finie $V_h\otimes V_h$ d'une structure d'espace normé de dimension finie. Le théorème de Bolzano-Weierstrass que $(g_n)$ admet une valeur d'adhérence $g_\infty$ dans $V_h\otimes V_h$.


Comme $r_n,s_n$ réalise l'optimum 
\end{ques}



\end{document}
