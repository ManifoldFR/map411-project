\documentclass[11pt]{article}
\usepackage{mathtools}
\usepackage{amssymb,amsthm,amsfonts}
\usepackage{titling}



\newcommand{\RR}{\mathbb R}
\newcommand{\CC}{\mathbb C}
\newcommand{\QQ}{\mathbb Q}
\newcommand{\ZZ}{\mathbb Z}
\newcommand{\NN}{\mathbb N}
\newcommand{\KK}{\mathbb K}
\newcommand{\FF}{\mathbb F}
\newcommand{\LL}{\mathbb L}

%%% Titling %%%

\pretitle{\hrulefill\begin{center}\LARGE}
\title{
	\textbf{Rapport de projet MAP411}:\\
	\textit{Résolution des problèmes elliptiques symétriques en grande dimension}}
\posttitle{\end{center}\hrulefill}


\author{
	\textit{Cheikh Fall}\\
	\textit{Wilson Jallet}\\
Promotion X2016}

%%% Questions %%%

\theoremstyle{definition}
\newtheorem{ques}{Question}

\setlength{\parindent}{0pt}

\begin{document}
\maketitle

\begin{ques}
On rappelle l'équation de Laplace
\begin{equation}\label{eqLapl}
\left\{
\begin{array}{l}
-\Delta u + u = f\ \text{sur}\ \Omega \\
\frac{\partial u}{\partial n} = 0\ \text{sur}\ \partial\Omega \\
u \in C^2(\overline{\Omega})
\end{array}
\right.
\end{equation}
où $\Omega = (0,1)^2$.

Supposons que la fonction $u:\overline\Omega\longrightarrow\RR$ soit une solution de \eqref{eqLapl}. Soit $v\in V$. Par une intégration par parties, on obtient
\begin{align*}
-\int_\Omega \Delta u\, v &= -\int_{\partial\Omega} \frac{\partial u}{\partial n}v + \int_\Omega \nabla u\cdot \nabla v \\
&= \int_\Omega \nabla u\cdot \nabla v \quad \text{d'après \eqref{eqLapl}}
\end{align*}
puis en utilisant que $-\Delta u = f-u$
\[
	-\int_\Omega uv + \int_\Omega fv = \int_\Omega \nabla u\cdot \nabla v
\]
et
\begin{equation}\label{laplVar}
\forall v\in V\quad
\int_\Omega \nabla u\cdot \nabla v + \int_\Omega uv = \int_\Omega fv.
\end{equation}


Réciproquement, si $u$ est une fonction de classe $C^2$ sur $\overline\Omega$ qui vérifie \eqref{laplVar} et $\partial u/\partial n = 0$ sur le bord de $\Omega$, une intégration par parties permet de retrouver 
\[
\forall v\in V\quad  -\int_\Omega \Delta u\, v + \int_\Omega uv = \int_\Omega fv
\]


\end{ques}





\end{document}