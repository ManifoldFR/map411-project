\documentclass[11pt]{article}
\usepackage[a4paper]{geometry}
\usepackage{mathtools}
\usepackage{amssymb,amsthm,amsfonts}
\usepackage{titling}
\usepackage{polyglossia}
\setdefaultlanguage{french}

\DeclareMathOperator{\id}{id}
\DeclareMathOperator{\Ker}{Ker}
\DeclareMathOperator{\Img}{Im}
\DeclareMathOperator{\Mat}{Mat}
\DeclareMathOperator{\Tr}{Tr}
\DeclareMathOperator{\rank}{rg}
\DeclareMathOperator{\vect}{Vect}
\DeclareMathOperator{\diagmat}{Diag}
\newcommand{\gl}{\mathrm{GL}}
\newcommand{\mev}[1]{\mathcal{M}_{#1}}
\newcommand{\dif}[1][{}]{\mathop{}\!\mathrm{d}^{#1}}

\newcommand{\NN}{\mathbb{N}}
\newcommand{\ZZ}{\mathbb{Z}}
\newcommand{\QQ}{\mathbb{Q}}
\newcommand{\RR}{\mathbb{R}}
\newcommand{\CC}{\mathbb{C}}
\newcommand{\KK}{\mathbb{K}}
\newcommand{\PP}{\mathbb{P}}
\DeclarePairedDelimiter{\zintcc}{[\![}{]\!]}
\DeclarePairedDelimiter{\zintco}{[\![}{[\![}
\DeclarePairedDelimiter{\zintoo}{]\!]}{[\![}
\DeclarePairedDelimiter{\zintoc}{]\!]}{]\!]}

%%% Titling %%%

\pretitle{\hrulefill\begin{center}\LARGE}
\title{
	\textbf{Rapport de projet MAP411}:\\
	\textit{Résolution des problèmes elliptiques symétriques en grande dimension}}
\posttitle{\end{center}\hrulefill}


\author{
	\textit{Cheikh Fall}\\
	\textit{Wilson Jallet}\\
Promotion X2016}

%%% Questions %%%

\theoremstyle{definition}
\newtheorem{ques}{Question}

%%% Font %%%
%\setmainfont{Times New Roman}

\setlength{\parindent}{0pt}

\begin{document}
\maketitle

\section{L'algorithme glouton}

\setlength{\parindent}{1cm}
\setlength{\parskip}{3mm}

\begin{ques}
On rappelle l'équation de Laplace
\begin{equation}\label{eqLapl}
\left\{
\begin{array}{l}
-\Delta u + u = f\ \text{sur}\ \Omega \\
\frac{\partial u}{\partial n} = 0\ \text{sur}\ \partial\Omega \\
u \in C^2(\overline{\Omega})
\end{array}
\right.
\end{equation}
où $\Omega = (0,1)^2$.

Supposons que la fonction $u:\overline\Omega\longrightarrow\RR$ soit une solution de \eqref{eqLapl}. Soit $v\in V$. Par une intégration par parties, on obtient
\begin{align*}
-\int_\Omega \Delta u\, v &= -\int_{\partial\Omega} \frac{\partial u}{\partial n}v + \int_\Omega \nabla u\cdot \nabla v \\
&= \int_\Omega \nabla u\cdot \nabla v \quad \text{d'après \eqref{eqLapl}}
\end{align*}
puis en utilisant que $-\Delta u = f-u$
\[
	-\int_\Omega uv + \int_\Omega fv = \int_\Omega \nabla u\cdot \nabla v
\]
et
\begin{equation}\label{laplVar}
\forall v\in V\quad
\int_\Omega \nabla u\cdot \nabla v + \int_\Omega uv = \int_\Omega fv.
\end{equation}


Réciproquement, si $u$ est une fonction de classe $C^2$ sur $\overline\Omega$ qui vérifie \eqref{laplVar} et $\partial u/\partial n = 0$ sur le bord de $\Omega$, une intégration par parties permet de retrouver 
\[
\forall v\in V\quad  -\int_\Omega \Delta u\, v + \int_\Omega uv = \int_\Omega fv
\]
donc
\[
\forall v\in V\quad \int_\Omega (-\Delta u + u -f)v = 0.
\]
D'après un lemme du cours, cela entraîne que\[ -\Delta u+ u -f = 0  \]
donc $u$ est solution de \eqref{eqLapl}.

\end{ques}


\begin{ques}
Pour tous $i$ et $j$ de $\zintcc{0,I}$,
\[
\nabla \phi_i\otimes\phi_j (x,y) = \left(
\phi_i'(x)\phi_j(y), \phi_i(x)\phi_j'(y)
\right)
\]
donc 
\begin{align*}
\int_{\Omega}\nabla(\phi_i\otimes\phi_j)\cdot\nabla(\phi_k\otimes\phi_l) 
&=
\int_{\Omega} \phi'_i(x)\phi_j(y)\phi_k'(x)\phi_l(y) + \phi_i(x)\phi_j'(y)\phi_k(x)\phi_l'(y)\,dx\,dy \\
&= \int_{0}^{1}\phi'_i\phi_k'\,dx\int_0^1\phi_j\phi_l\,dy + \int_{0}^{1}\phi_i\phi_k\,dx\int_0^1\phi_j'\phi_l'\,dy\ \text{(théorème de Fubini)} \\
&= D_{i,k}M_{j,l} + M_{i,k}D_{j,l}.
\end{align*}

Ainsi, avec $u_h = \sum_{i,j=0}^{I}U_{i,j}\phi_i\otimes\phi_j$ on obtient
\begin{align*}\tag{a}\label{eq:a}
\int_{\Omega} \nabla u_h(x,y)\cdot \nabla \phi_k\otimes\phi_l (x,y)\,dx\,dy 
&= \sum_{i,j=0}^{I}U_{i,j} \int_\Omega\nabla(\phi_i\otimes\phi_j)\cdot\nabla(\phi_k\otimes\phi_l) \\
&= \sum_{i,j=0}^{I}U_{i,j} (D_{i,k}M_{j,l}+M_{i,k}D_{j,l}).
\end{align*}
On a de plus
\[
\int_\Omega (\phi_i\otimes\phi_j)(x,y)(\phi_k\otimes\phi_l)(x,y)\,dx\,dy = \int_0^1 \phi_i \phi_k\,dx \int_0^1 \phi_j \phi_l\,dy = M_{i,k}M_{k,l}
\]
d'où
\begin{equation}\tag{b}\label{eq:b}
\int_\Omega u_h(x,y)\phi_k\otimes\phi_l = \sum_{i,j=0}^{I}U_{i,j}M_{i,k}M_{j,l}
\end{equation}

En supposant que $u_h$ est une solution de \eqref{laplVar}, sommer \eqref{eq:a} et \eqref{eq:b} donne
\begin{align}\label{laplDiscr}
\forall (k,l)\in \zintcc{0,I}^2\quad
\sum_{i,j=0}^{I}U_{i,j}(D_{i,k}M_{j,l} + M_{i,k}D_{j,l} + M_{i,k}M_{j,l}) &=
\int_\Omega\nabla u_h\cdot \nabla\phi_k\otimes\phi_l + \int_\Omega u_h (\phi_k\otimes\phi_l) \nonumber \\
&= F_{k,l}.
\end{align}


Réciproquement, si le vecteur $U$ vérifie le système linéaire \eqref{laplDiscr}, alors la fonction $u_h = \sum_{i,j=0}^{I}U_{i,j}\phi_i\otimes\phi_j$ vérifie \eqref{laplVar} pour $v$ dans l'espace engendré par les $\phi_i\otimes\phi_j$.

Montrons que \eqref{laplDiscr} est bien posé. On le réécrit sous la forme
\begin{equation}\tag{c}\label{laplComp}
T(X) = F
\end{equation}
où $T$ est un endomorphisme sur l'espace $E=\RR^{(I+1)\times(I+1)}$, dont la matrice est
\[
T_{k,l}^{i,j} = D_{i,k}M_{j,l} + M_{i,k}D_{j,l} + M_{i,k}M_{j,l}
\]
ce qui montre que $T$ est symétrique. Il est positif pour le produit scalaire $\langle X,Y\rangle = \Tr(X^tY)$ : si $X\in E$ et $f = \sum_{i,j=0}X_{i,j}\phi_i\otimes\phi_j$, on a d'après les calculs précédents
\begin{align*}
\langle X,T(X)\rangle &= \sum_{k,l=0}^IX_{k,l}\sum_{i,j=0}^IX_{i,j}T_{k,l}^{i,j} 
=
\sum_{k,l=0}^IX_{k,l}\left[\int_\Omega \nabla f\cdot \nabla \phi_i\otimes\phi_j + \int_\Omega f(\phi_i\otimes\phi_j) \right] \\
&= \int_\Omega |\nabla f|^2 + \int_\Omega f^2 \geq 0,
\end{align*}
et l'égalité entraîne par continuité de $f$ que $f = 0$ puis $X = 0$ par liberté de la famille $(\phi_i\otimes\phi_j)_{0\leq i,j\leq I}$. On en déduit donc que $T:E\rightarrow E$ est injectif, puis est un isomorphisme.

Si on travaille sur un pavé $\Omega = (0,1)^d$ en dimension $d$ avec la même méthode, la base tensorielle est $\phi_{i_1}\otimes\cdots\otimes\phi_{i_d}$ et on travaille avec des données $U \in \RR^{(I+1)^d}$.

\end{ques}


\begin{ques}
Si $U\in E=\RR^{(I+1)^2}$ et $u_h = \sum_{i,j=0}^I\phi_i\otimes\phi_j$, remarquons qu'avec les notations introduites dans \eqref{laplComp}
\[
\mathcal{E}(u_h) = \frac 12\int_\Omega |\nabla u_h|^2 + \frac 12\int_\Omega u_h^2 - \int_\Omega fu_h = \frac 12\langle U,T(U)\rangle - \langle U,F\rangle.
\]

Le membre de droite est une application de classe $C^1$ sur l'espace $E$ dont le gradient en tout point $U$ est
\[
T(U) - F
\]
\end{ques}


\begin{ques}

\end{ques}



\section{Équations d'Euler et convergence de l'algorithme}








\end{document}